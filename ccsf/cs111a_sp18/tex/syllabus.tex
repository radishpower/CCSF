% This syllabus template was created by:
% Brian R. Hall
% Associate Professor, Champlain College
% www.brianrhall.net

% Document settings
\documentclass[11pt]{article}
\usepackage[margin=1in]{geometry}
\usepackage[pdftex]{graphicx}
\usepackage{multirow}
\usepackage{setspace}
\pagestyle{plain}
\setlength\parindent{0pt}

\begin{document}

% Course information
\begin{tabular}{ l l }
  \multirow{3}{*}{\includegraphics[height=1.25in,width=1.25in]{logo_ccsf.png}} & \LARGE CS111A \\\\
                                                                             & \LARGE Introduction to Java \\\\
  & \LARGE Mon/Wed 9:10AM to 11AM \\\\
  & \LARGE Location: ACRC, PC-Lab 1\\\\
\end{tabular}
\vspace{10mm}

% Professor information
\begin{tabular}{ l l }
  % \multirow{6}{*}{\includegraphics[height=1.25in,width=1in]{logo_ccsf.png}} & \large Grace Woo \\\\
  & \large grwoo@ccsf.edu \\
  & \large Canvas Website: TBD \\
  & \large Office Hours:  Mondays 3:30pm to 5pm in Batmale 456\\
\end{tabular}
\vspace{5mm}
\begin{center} Subject to changes throughout the semester.\\
\end{center}

% Course details
\textbf {\large \\ Course Description:} This course is a first course in computer programming. You are expected to know how to use a computer for basic tasks including e-mail and browsing the world wide web, but no futher background in computers is assumed. The emphasis will be on principals of computer programming, using the computer programming language Java. We will use Java as a tool that enables us to study computer programming, so you will learn many important features of Java. Thus, after completing this course you will be prepared to go on to study other programming languages or continue to study Java.\\

Learning to write computer programs is a time consuming and sometimes frustrating endeavor. I expect an average student to spend about 8 hours per week outside of class reading and working on programming assignments and other class work. If you don't have the time or dedication for such work, this class may not be for you.\\

\textbf {Attendance Policy:} You are expected to attend all classes and be
seated for the class to begin promptly at ten minutes after the hour, when roll
will be taken. Participation accounts for 10\% of your grade. Any information
that you miss due to nonattendance is solely your responsibility. This may
include helpful information for assignments and tests. {\bf I may drop you from
the class if you miss more than 3 classes in a row without explanation.}\\

\textbf {Prerequisite(s):} None.

\textbf {Credit Hours:} 4 units\\
\textbf {Days:} Monday and Wednesdays: 9:10 to 11am in PC-Lab 1 of the ACRC, Batmale 301 (Ocean/Phelan campus)
Class will begin promptly at 9:10. There will be a ten-minute break at 10:00 if we're doing an in-class exercise.\\
\textbf {\large Text(s):} \emph{Java Concepts: Late Objects}, 3\textsuperscript{rd} Edition\\
\textbf {Author(s):} Horstmann;\\
\textbf {ISBN:} 978-1-119-32102-6 \\

\textbf {\large Student Learning Outcome(s):} \\
At the completion of this course, students will be able to:
\begin{enumerate} \itemsep-0.4em
  \item Describe the software development life-cycle and the use of algorithms in program design
  \item Develop, implement and accurately predict the results of structured programs and code in Java, including the use of numeric and Boolean expressions, if and switch statements, loops and nested control structures
  \item Write Java code with, and accurately predict the results of, methods that have reference and value parameters and return values
  \item Write Java code to pass and process arrays and Strings
\end{enumerate}

% I recommend using \newpage here if necessary
\textbf {\large Grading Policy:} \\
\hspace*{40mm}
\begin{tabular}{ l l }
Participation in excercises and activities during class time & 10\% \\
Programming Labs (10 pts / each)& 40\% \\
Test \#1 (25 pts) & 10\% \\
Test \#2 (25 pts) & 10\% \\
Test \#3  (25 pts) & 10\% \\
Final Exam  (50 pts)& 20\%
\end{tabular} \\\\

There will be three tests given in this class, in addition to the final exam.
The first two and the assignments graded in time will be the basis for the
midterm grade. If you will not be able to take a test when it's scheduled, you
must notify me a few days before the test to request a make-up time.\\

\textbf {\large Midterm and final grades will be assigned on the following percentage scale:} \\\\
\hspace*{40mm}
\begin{tabular}{ l l | l l }
90\% - 100\% A \\
80\% - 89\% B \\
70\% - 79\% C \\
60\% - 69\% D \\
0\% - 59\% F \\
\end{tabular} \\

Students who do not take the final exam will be assigned a grade of "FW". An "FW" is an "F" grade that also indicates that the student did not complete the course.\\

% Course Policies. These are just examples, modify to your liking.

\textbf {\large Homework:}

The best way to learn how to program is to do it! Homework will be assigned
about once a week, generally alternating between Programming Labs and Practice
Problems.

\begin{itemize}
	\item \textbf {Programming Labs:}

          You must write and understand the assignment yourself. If you collaborate with other students, clearly indicate who in your comments. Programming Labs will be graded based upon their correctness, clarity, and programming style.

          Homework will usually be due on Wednesday nights at midnight, but you should try to complete them early, so you can ask questions and get help. You will each encounter problems that require more time than you anticipate to fix -- that's the nature of programming. So think of the homeworks as due before class. Then you can ask questions in class on the due date if necessary.

          All homework submissions must include 2 major parts: the Java code you wrote (the source file) and some sample input and output showing how your program works. To turn in a problem that you may have worked on in a group, each team member should submit the assignment, listing whom you work with.

          I will employ a student worker to grade homework assignments for this class. If you have any questions or concerns about this arrangements or a particular grading decision the grader makes, please don't hesitate to tell me. I will be happy to review grading decisions on request.

\end{itemize}

\textbf {Homework Lateness policy:}

\hspace{3mm}

        Because of the importance of keeping up with the pace of class, late
        homework will be penalized severely. All homework assignments are due
        by midnight the night of the due date specified. Late homework will be
        penalized 5\% if it is turned in before I go through the solution in
        class (the following class after it's due). Starting the day I present
        the solution, late homework will be penalized 50\%. You will get no
        credit for turning in my solution as your own. All homework you turn in
        must be your own, even after we have gone through a solution in class.

\hspace{3mm}

\textbf{Cheating}

\hspace{3mm}

Cheating of any kind will not be tolerated. It will result in a grade of 0 on
the assignment or test in question and can be cause for a failed grade and
disciplinary action, including suspension or expulsion. Cheating on Programming
Labs means copying code or answers from someone else. Getting help from others
is not cheating as long as you're not copying their work or allowing them to
copy yours. On the exams, any collaboration or copying constitutes cheating.

\hspace{3mm}

\textbf{Software and Computer Access}

\hspace{3mm}

Classroom assignments can be completed on repl.it before being copy and pasted
into Canvas for submission. I encourage you to use Oracle's Standard Edition
(SE) Java Development Kit: Java SE JDK 8. It is already installed on the CCSF
Linux and Windows systems. That means all your homework can be done on your own
computer or using the City College Linux server ``hills''. By registering in
this class you will automatically be given an account on hills, or if you
already had an account, it will be reactivated if necessary. You can access
hills from any computer that is connected to the Internet.\\

For more information about how to use the CCSF computer systems for your
classwork, see Craig Persiko's Computer Access and Use Information handout.\\

Use of CCSF computers, including remote access, is regulated by the CCSF
Computer Usage Policy, which is found in hte college catalogue and on the web
at http://www.ccsf.edu/Policy/policy.shtml. Do not give passwords and other
sensitive information to unauthorized persons. This means you shouldn't tell
anyone your personal passwords and you shouldn't give class account passwords
to people who aren't in this class.

\hspace{3mm}

\textbf{Drop Procedures}

\hspace{3mm}

Generally it is your responsibility to drop or withdraw from a class by the
final deadlines given in your course schedule. Do not ask me to drop you; use
the Web4 system, or contact the Office of Admissions and Records to be
withdrawn from a class. If you have more than three unexplained consecutive
class absences, I may drop you from the class. If your name is on the roll at
the end of the semester and you have stopped attending class, you will be
assigned a final grade of FW. I will not give a late or retroactive drop or
withdrawal.

\hspace{3mm}

\textbf{Disability Accomodations}

\hspace{3mm}

Students with disabilities who need accommodations are encouraged to contact
the instructor. Disabled Students Programs and Services (DSPS) is available to
facilitate the reasonable accommodation process. The DSPS office is located in
the Rosenberg Library, Room 323 and can be reached at (415) 452-5481.

% Course Outline

\hspace{3mm}

\textbf {\large Tentative Course Outline}:

The weekly coverage might change as it depends on the progress of the class.
However, you must keep up with the reading assignments.

\begin{table}[h!]
  \normalsize % The size of the table text can be changed depending on content. Remove if desired.

\begin{tabular}{ | c | c | }
\hline
\textbf{Due Date} & \textbf{Homework Schedule} \\
\hline
1/22 & \begin{minipage}{.85\textwidth}
\begin{itemize} \itemsep-0.4em
	\vspace{1mm}
	\item Programming Lab 1 Due: Hour of Code Programming Puzzles
	\vspace{1mm}
\end{itemize}
\end{minipage} \\
\hline
1/24 & \begin{minipage}{.85\textwidth}
\begin{itemize} \itemsep-0.4em
	\vspace{1mm}
	\item Programming Lab 2 Due: Hello World
	\vspace{1mm}
\end{itemize}
\end{minipage} \\
\hline
1/31 & \begin{minipage}{.85\textwidth}
\begin{itemize} \itemsep-0.4em
	\vspace{1mm}
	\item Programming Lab 3 Due: Muni Ridership Calculator
	\vspace{1mm}
\end{itemize}
\end{minipage} \\
\hline
2/7 & \begin{minipage}{.85\textwidth}
\begin{itemize} \itemsep-0.4em
	\vspace{1mm}
      \item Test \#1 in class: Covers Chapters 1 (Introduction) and 2 (Fundamentals), and the first part of Chapter 3 (If-Statements) 
	\vspace{1mm}
\end{itemize}
\end{minipage} \\
\hline
2/14 & \begin{minipage}{.85\textwidth}
\begin{itemize} \itemsep-0.4em
	\vspace{1mm}
	\item Programming Lab 4 Due: Time Calculator
	\vspace{1mm}
\end{itemize}
\end{minipage} \\
\hline
2/21 & \begin{minipage}{.85\textwidth}
\begin{itemize} \itemsep-0.4em
	\vspace{1mm}
        \item No class on 2/19 (President's Day) 
	\item Reading assignment
	\vspace{1mm}
\end{itemize}
\end{minipage} \\
\hline
2/28 & \begin{minipage}{.85\textwidth}
\begin{itemize} \itemsep-0.4em
	\vspace{1mm}
	\item Programming Lab 5 Due: Rock - Paper - Scissors Game
	\vspace{1mm}
\end{itemize}
\end{minipage} \\
\hline
3/7 & \begin{minipage}{.85\textwidth}
\begin{itemize} \itemsep-0.4em
	\vspace{1mm}
	\item Programming Lab 6 Due: Jackalope Populations
	\vspace{1mm}
\end{itemize}
\end{minipage} \\
\hline
3/14 & \begin{minipage}{.85\textwidth}
\begin{itemize} \itemsep-0.4em
	\vspace{1mm}
	\item Programming Lab 7 Due: Parallelogram Program
	\vspace{1mm}
\end{itemize}
\end{minipage} \\
\hline
3/21 & \begin{minipage}{.85\textwidth}
\begin{itemize} \itemsep-0.4em
	\vspace{1mm}
        \item Test \#2 in class: Covers Chapters 3 (Decisions) and 4 (Loops) 
	\vspace{1mm}
\end{itemize}
\end{minipage} \\
\hline
3/28 & \begin{minipage}{.85\textwidth}
\begin{itemize} \itemsep-0.4em
	\vspace{1mm}
	\item Spring Break: No classes 3/26 - 3/28
	\vspace{1mm}
\end{itemize}
\end{minipage} \\
\hline
4/4 & \begin{minipage}{.85\textwidth}
\begin{itemize} \itemsep-0.4em
	\vspace{1mm}
	\item Programming Lab 8 Due: Consumer Loan Program
	\vspace{1mm}
\end{itemize}
\end{minipage} \\
\hline
4/11 & \begin{minipage}{.85\textwidth}
\begin{itemize} \itemsep-0.4em
	\vspace{1mm}
	\item Programming Lab 9 Due: Analyze Phrase
	\vspace{1mm}
\end{itemize}
\end{minipage} \\
\hline
4/18 & \begin{minipage}{.85\textwidth}
\begin{itemize} \itemsep-0.4em
	\vspace{1mm}
        \item Programming Lab 10 Due: Palindrome Program
        \item Test \#3 in class: Covers Chapters 5 (Methods), Chapter 2 (String Manipulation) and Chapter 6 (Arrays)
	\vspace{1mm}
\end{itemize}
\end{minipage} \\
\hline
4/25 & \begin{minipage}{.85\textwidth}
\begin{itemize} \itemsep-0.4em
        \vspace{1mm}
        \item Programming Lab 11 Due: De-Dup Program
        \vspace{1mm}
\end{itemize}
\end{minipage} \\
\hline
5/2 & \begin{minipage}{.85\textwidth}
\begin{itemize} \itemsep-0.4em
        \vspace{1mm}
        \item Programming Lab 12 Due: Distance File
        \vspace{1mm}
\end{itemize}
\end{minipage} \\
\hline
5/9 & \begin{minipage}{.85\textwidth}
\begin{itemize} \itemsep-0.4em
        \vspace{1mm}
        \item Extra Credit: Design a Lottery
        \vspace{1mm}
\end{itemize}
\end{minipage} \\
\hline
5/16 & \begin{minipage}{.85\textwidth}
\begin{itemize} \itemsep-0.4em
        \vspace{1mm}
        \item Final Exam
        \vspace{1mm}
\end{itemize}
\end{minipage} \\
\hline
\end{tabular} 
\end{table}

\end{document}




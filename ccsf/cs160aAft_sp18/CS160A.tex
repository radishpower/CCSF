% This syllabus template was created by:
% Brian R. Hall
% Associate Professor, Champlain College
% www.brianrhall.net

% Document settings
\documentclass[11pt]{article}
\usepackage[margin=1in]{geometry}
\usepackage[pdftex]{graphicx}
\usepackage{multirow}
\usepackage{setspace}
\pagestyle{plain}
\setlength\parindent{0pt}

\begin{document}

% Course information
\begin{tabular}{ l l }
  \multirow{3}{*}{\includegraphics[height=1.25in,width=1.25in]{logo_ccsf.png}} & \LARGE CS160A \\\\
                                                                             & \LARGE Introduction to Linux \\\\
  & \LARGE Mon/Wed 1:10PM to 3PM \\\\
  & \LARGE Location: Batmale 413\\\\
\end{tabular}
\vspace{10mm}

% Professor information
\begin{tabular}{ l l }
  % \multirow{6}{*}{\includegraphics[height=1.25in,width=1in]{logo_ccsf.png}} & \large Grace Woo \\\\
  & \large grwoo@ccsf.edu \\
  & \large Canvas Website: TBD \\
  & \large Office Hours:  Mondays 3:30pm to 5pm in Batmale 456\\
\end{tabular}
\vspace{5mm}
\begin{center} Subject to changes throughout the semester.\\
\end{center}

% Course details
\textbf {\large \\ Course Description:}

This is a course in basic Linux. As such, the skills learned in this course
will be applicable to any Linux or Unix system or variant. The primary shell we
use in the bash shell. The shell constructs taught, however, are applicable to
any POSIX-compliant shell. Major topics covered include navigating, the
command-line structure of Linux, Linux text processing tools, regular
expressions, use of the shell, file manipulation commands and permissions.
This course is a significant commitment of time. You should plan to spend an
average of 2-5 hours of study and lab time for every hour spent in class for
the duration of this course.\\

\textbf {Attendance Policy:} You are expected to attend all classes and be
seated for the class to begin promptly at ten minutes after the hour, when roll
will be taken. Any information
that you miss due to nonattendance is solely your responsibility. This may
include helpful information for assignments and tests. {\bf I may drop you from
the class if you miss more than 2 classes in a row without explanation.}\\

\textbf {Prerequisite(s):} None.

\textbf {Credit Hours:} 2 units\\
\textbf {Days:} Monday and Wednesdays: 1:10 to 3pm in Batmale 413 (Ocean/Phelan campus)
Class will begin promptly at 1:10. There will be a ten-minute break at 2:00 if we're doing an in-class exercise.\\
\textbf {\large Text(s):} \emph{The Linux Command Line}, Available Online for Free\\
\textbf {Author(s):} William Shotts;\\
\textbf {URL:} \url{http://linuxcommand.org/tlcl.php} \\

\textbf {\large Student Learning Outcome(s):} \\
At the completion of this course, students will be able to:
\begin{enumerate} \itemsep-0.4em
  \item Create and use Unix file management utilities to manipulate files and directories.
  \item Write and predict the effects of commands that manipulate file and directory permissions.
  \item Employ text file utilities to display, sort, replace and edit text.
  \item Write and predict the results of Unix commands that use regular expressions to search for patterns in files.
\end{enumerate}

% I recommend using \newpage here if necessary
\textbf {\large Grading Policy:} \\
\hspace*{40mm}
\begin{tabular}{ l l }
  Assignments: 90pts \\
  Labs: 20pts \\
  Quizzes: 70pts \\
  Final Examination: 70pts \\
\end{tabular} \\\\

There will be two midterms (quizzes) during the semester, focusing on recent
material. You may miss one midterm, but if you do, to make up for it, the next
midterm counts double or the weight of the final is increased proportionally.
There will also be a required cumulative final examination. One standard sheet
of paper containing notes on both sides may be brought to midterms. Three
standard sheets of paper containing notes may be brought to the final exam. No
electronic devices are allowed.\\

\textbf {\large Midterm and final grades will be assigned on the following percentage scale:} \\\\
\hspace*{40mm}
\begin{tabular}{ l l | l l }
90\% - 100\% A \\
80\% - 89\% B \\
70\% - 79\% C \\
60\% - 69\% D \\
0\% - 59\% F \\
\end{tabular} \\

Students who do not take the final exam will be assigned a grade of "FW". An "FW" is an "F" grade that also indicates that the student did not complete the course.\\

% Course Policies. These are just examples, modify to your liking.

\textbf {\large Homework:}

The best way to learn how to program is to do it!

\begin{itemize}
	\item \textbf {Exercise Sets:}

A significant number of required additional ungraded exercise sets will be
assigned during this course. Although you do not hand these in, they are very
important and comprise the meat of the course. Some will have solutions, others
are meant for you to do on the machine and test for yourself. Do them! The
ungraded exercise sets are the primary source of questions for tests. The
web-version of the class syllabus has links to PDFs of the exercise sets.

	\item \textbf {Taking Notes:}

The act of taking notes helps remember things to come back to later. I do not give out notes on lectures. If you are not in class, it is your responsibility to get the information from a classmate.

        \item \textbf {Assignments and Labs:}

There will be 3 graded assignments during the course. They will be assigned 1-2
weeks prior to their due date.  Assignments must be handed in in-class at the
beginning of class on the due date. Assignments are not accepted late for full
credit. If you turn in an assignment late, it is kept until the end of the
semester. At that time it will be considered for a maximum of 50\% credit.
Assignments MUST be done on the assigned CCSF system.  Specific requirements
for assignments are covered in the handout entitled AssignmentGuidelines.
There will also be 7 lab exercises. Five of them are in-class. They count 2 pts
each. Four count in your grade, The remaining one is extra-credit. The other
two labs are take-home labs. They must be turned in at the next class and count
5 points each. Labs cannot be made up.

\end{itemize}

\textbf {Group Work:}

\hspace{3mm}

You may work together with one other student as a 'group'. If you work on an
assignment as a group, hand in a single copy of the assignment with both
contributors' names, login, day and time of class on the front. Both
contributors will receive the same grade. If you want to work together, hand in
a single copy as described above. Otherwise, if I receive two identical
assignments I cannot grade them. 

\hspace{3mm}

\textbf{Getting Help:}

\hspace{3mm}

This course has a Google Group that you must join. Postings to the group earn
extra-credit points at the end of the semester. Email the list with questions
that come up between class sessions. The ACRC lab (3rd floor Batmale hall) also
has tutors who have are willing to help with basic Linux. Ask in the lab. I try
to make myself as available as possible to my students. 

\hspace{3mm}

\textbf{Hills Accounts:}

\hspace{3mm}

Within a few days of completing your enrollment for this class, an account will
be created for you on our school Linux server (hills). If you are adding, it is
important that you complete your add as soon as possible. Delaying your add to
the class will not be considered an excuse for turning in assignments late. If
you had a hills account last semester, your account and password will remain
the same. If you are getting a new hills account, your account name will be the
same as your CCSF e-mail account (if you did not "opt-out"). Your initial
password is formed from your birth date - combining the first three letters of
the month (lowercase) with the two-digit day and the two-digit year followed by
a period and the first two characters of your login; e.g., if your e-mail
account is scharo11@mail.ccsf.edu, and your birthday is Apr 14 1986 your hills
login would be scharo11 and your initial password would be apr1486.sc You
should immediately change your initial password.  Class data files are on hills
in the public work area at /pub/cs/gboyd/cs160a

\hspace{3mm}

\textbf{Access to Hills:}

\hspace{3mm}

You can access hills either from a computer in the ACRC in Batmale Hall or
remotely using ssh. If you access hills from the ACRC you should use the linux
machines near the rear exit (see the next section). You may also login from a
Windows system, but you must first login to the ACRC Windows network. If you
wish to do this, you should take an orientation during the first week of class.
You can access hills remotely using ssh. Do not use telnet. The particulars of
remote access are your responsibility.  The server is hills.ccsf.edu. You are
also responsible for figuring out how to print your assignments. It is your
responsibility to get these issues worked out in order to complete your
assignments on time.  The ACRC holds a series of three orientation classes on
hills and on their Windows network. A schedule is posted in the ACRC. If you
are new to hills or the ACRC you should consider attending these sessions.

\hspace{3mm}

\textbf{Access to Linux Machines (Springfield Cluster)}

\hspace{3mm}

By enrolling in this course you will have an account on the linux machines.
These accounts will be created a week or so into the semester as announced in
class. The account name and initial password follow the same pattern as your
hills account. The linux machines are divided between those in the ACRC and
those in the linux classroom. They all share a common set of logins and a
common exported file system for home directories. Once on linux, you can use
ssh at the command-line to log in to hills to access the class public data
files.  For security reasons the linux machines are only accessible through
hills or another local machine. They are not registered via DNS. If you want to
reach a linux system from off-campus, you must login to hills and ssh using the
IP address of a linux machine. These IP addresses are taped to the linux
machines in the linux area of the ACRC. You should visit it and make a note of
them.

% Course Outline

\hspace{3mm}

\textbf {\large Tentative Course Outline}:

The weekly coverage might change as it depends on the progress of the class.
However, you must keep up with the reading assignments.

\begin{table}[h!]
  \normalsize % The size of the table text can be changed depending on content. Remove if desired.

\begin{tabular}{ | c | c | }
\hline
\textbf{Due Date} & \textbf{Homework Schedule} \\
\hline
8/21 & \begin{minipage}{.85\textwidth}
\begin{itemize} \itemsep-0.4em
        \vspace{1mm}
        \item Lab 1 Exercise
        \item Logging in: console and \bf{ssh}
        \item Introduction to commands and redirection
        \item Simple commands: \bf{cp, mv, cat, less, file, ls} (with options -F, -l, -a)
        \vspace{1mm}
\end{itemize}
\end{minipage} \\
\hline
8/23 & \begin{minipage}{.85\textwidth}
\begin{itemize} \itemsep-0.4em
	\vspace{1mm}
        \item Paths Exercise 
	\item Creating files using redirection
        \item Navigating directory structures
	\vspace{1mm}
\end{itemize}
\end{minipage} \\
\hline
8/28 & \begin{minipage}{.85\textwidth}
\begin{itemize} \itemsep-0.4em
        \vspace{1mm}
        \item Wildcards Exercise
        \item SCP/SFTP Exercise
        \vspace{1mm}
\end{itemize}
\end{minipage} \\
\hline
8/30 & \begin{minipage}{.85\textwidth}
\begin{itemize} \itemsep-0.4em
	\vspace{1mm}
	\item Lab 2 Exercise
	\vspace{1mm}
\end{itemize}
\end{minipage} \\
\hline
9/1 & \begin{minipage}{.85\textwidth}
\begin{itemize} \itemsep-0.4em
	\vspace{1mm}
        \item File Utilities 1 Exercise
        \item File Utilities 2 Exercise
	\vspace{1mm}
\end{itemize}
\end{minipage} \\
\hline
9/6 & \begin{minipage}{.85\textwidth}
\begin{itemize} \itemsep-0.4em
	\vspace{1mm}
        \item Assignment 1 DUE
        \item No class on 9/4 ( Labor Day )
        \item Find Exercise
        \item Lab 3 Exercise
        \item Practice Permissions
	\vspace{1mm}
\end{itemize}
\end{minipage} \\
\hline
9/11 & \begin{minipage}{.85\textwidth}
\begin{itemize} \itemsep-0.4em
        \vspace{1mm}
      \item Quiz 1 IN CLASS
        \vspace{1mm}
\end{itemize}
\end{minipage} \\
\hline
9/13 & \begin{minipage}{.85\textwidth}
\begin{itemize} \itemsep-0.4em
	\vspace{1mm}
        \item Permissions Exercise
        \item Lab 4 Exercise (Take Home)
	\vspace{1mm}
\end{itemize}
\end{minipage} \\
\hline
9/18 & \begin{minipage}{.85\textwidth}
\begin{itemize} \itemsep-0.4em
        \vspace{1mm}
        \item File Utilities 3 Exercise
        \item Filters 1 Exercise
        \vspace{1mm}
\end{itemize}
\end{minipage} \\
\hline
9/20 & \begin{minipage}{.85\textwidth}
\begin{itemize} \itemsep-0.4em
        \vspace{1mm}
        \item Lab 5 Exercise
	\vspace{1mm}
\end{itemize}
\end{minipage} \\
\hline
9/25 & \begin{minipage}{.85\textwidth}
\begin{itemize} \itemsep-0.4em
	\vspace{1mm}
        \item Quiz 2 IN CLASS
	\vspace{1mm}
\end{itemize}
\end{minipage} \\
\hline
9/27 & \begin{minipage}{.85\textwidth}
\begin{itemize} \itemsep-0.4em
	\vspace{1mm}
        \item Assignment 2 DUE
	\item Filters 2 Exercise
        \item Lab 6 Exercise (Take Home)
	\vspace{1mm}
\end{itemize}
\end{minipage} \\
\hline
10/2 & \begin{minipage}{.85\textwidth}
\begin{itemize} \itemsep-0.4em
	\vspace{1mm}
	\item Basic Regular Expressions Exercise
	\vspace{1mm}
\end{itemize}
\end{minipage} \\
\hline
10/4 & \begin{minipage}{.85\textwidth}
\begin{itemize} \itemsep-0.4em
	\vspace{1mm}
	\item Sed Exercise
	\vspace{1mm}
\end{itemize}
\end{minipage} \\
\hline

10/11 & \begin{minipage}{.85\textwidth}
\begin{itemize} \itemsep-0.4em
	\vspace{1mm}
        \item No class on 10/9 ( Flex Day )
        \item Extended Regular Expressions Exercise
	\vspace{1mm}
\end{itemize}
\end{minipage} \\
\hline

10/16 & \begin{minipage}{.85\textwidth}
\begin{itemize} \itemsep-0.4em
	\vspace{1mm}
        \item Lab 7 Exercise
	\vspace{1mm}
\end{itemize}
\end{minipage} \\
\hline

10/18 & \begin{minipage}{.85\textwidth}
\begin{itemize} \itemsep-0.4em
	\vspace{1mm}
        \item Assignment 3 DUE 
	\item FINAL EXAM
	\vspace{1mm}
\end{itemize}
\end{minipage} \\
\hline
\end{tabular} 
\end{table}

\end{document}

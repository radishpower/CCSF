% This syllabus template was created by:
% Brian R. Hall
% Associate Professor, Champlain College
% www.brianrhall.net

% Document settings
\documentclass[11pt]{article}
\usepackage[margin=1in]{geometry}
\usepackage[pdftex]{graphicx}
\usepackage{multirow}
\usepackage{setspace}
\pagestyle{plain}
\setlength\parindent{0pt}

\begin{document}

% Course information
\begin{tabular}{ l l }
  \multirow{3}{*}{\includegraphics[height=1.25in,width=1.25in]{logo_ccsf.png}} & \LARGE CS160B \\\\
                                                                             & \LARGE Introduction to Linux \\\\
  & \LARGE Mon 6:10PM to 10PM \\\\
  & \LARGE Location: Batmale 413\\\\
\end{tabular}
\vspace{10mm}

% Professor information
\begin{tabular}{ l l }
  % \multirow{6}{*}{\includegraphics[height=1.25in,width=1in]{logo_ccsf.png}} & \large Grace Woo \\\\
  & \large Contact: grwoo@ccsf.edu \\
  & \large Office Hours: By arrangement\\
\end{tabular}
\vspace{5mm}
\begin{center} Subject to changes throughout the semester.\\
\end{center}

% Course details
\textbf {\large \\ Course Description:}

This is a course in basic Linux. As such, the skills learned in this course
will be applicable to any Linux or Unix system or variant. The primary shell we
use in the bash shell. The shell constructs taught, however, are applicable to
any POSIX-compliant shell. Major topics covered include navigating, the
command-line structure of Linux, Linux text processing tools, regular
expressions, use of the shell, file manipulation commands and permissions.
This course is a significant commitment of time. You should plan to spend an
average of 2-5 hours of study and lab time for every hour spent in class for
the duration of this course.\\

\textbf {Attendance Policy:} You are expected to attend all classes and be
seated for the class to begin promptly at ten minutes after the hour, when roll
will be taken. Any information
that you miss due to nonattendance is solely your responsibility. This may
include helpful information for assignments and tests. {\bf I may drop you from
the class if you miss more than 2 classes in a row without explanation.}\\

\textbf {Prerequisite(s):} None.

\textbf {Credit Hours:} 2 units\\
\textbf {Days:} Mondays: 6:10 to 10pm in Batmale 413 (Ocean/Phelan campus)
Class will begin promptly at 6:10. There will be at least two 5 minute breaks during these sessions.\\
\textbf {\large Text(s):} \emph{The Linux Command Line}, Available Online for Free\\
\textbf {Author(s):} William Shotts;\\
\textbf {URL:} \url{http://linuxcommand.org/tlcl.php} \\

\textbf {\large Student Learning Outcome(s):} \\
At the completion of this course, students will be able to:
\begin{enumerate} \itemsep-0.4em
  \item Create and use Unix file management utilities to manipulate files and directories.
  \item Write and predict the effects of commands that manipulate file and directory permissions.
  \item Employ text file utilities to display, sort, replace and edit text.
  \item Write and predict the results of Unix commands that use regular expressions to search for patterns in files.
\end{enumerate}

% I recommend using \newpage here if necessary
\textbf {\large Grading Policy:} \\
\hspace*{40mm}
\begin{tabular}{ l l }
  Assignments (10 pts each): 30pts \\
  In-Class Participation: 70pts \\
  Quizzes (25 pts each): 50pts \\
  Final Examination: 50pts \\
\end{tabular} \\\\

There will be two midterms (quizzes) during the semester, focusing on recent
material. You may miss one midterm, but if you do, to make up for it, the next
midterm counts double or the weight of the final is increased proportionally.
There will also be a required cumulative final examination. Notes as well as
online access are allowed for all quizzes. Likewise, online access is allowed
for the final exam.\\

\textbf {\large Midterm and final grades will be assigned on the following percentage scale:} \\\\
\hspace*{40mm}
\begin{tabular}{ l l | l l }
90\% - 100\% A \\
80\% - 89\% B \\
70\% - 79\% C \\
60\% - 69\% D \\
0\% - 59\% F \\
\end{tabular} \\

Students who do not take the final exam will be assigned a grade of "FW". An "FW" is an "F" grade that also indicates that the student did not complete the course.\\

% Course Policies. These are just examples, modify to your liking.

\textbf {\large Homework:}

The best way to learn how to program is to do it!

\begin{itemize}
	\item \textbf {Exercise Sets:}

In-class exercise sets and checkoffs will be used to determine participation points. Although these will not be handed in, they are
important and comprise the meat of the course. Some will have solutions, others
are meant for you to do on the machine and test for yourself. Do them! The
ungraded exercise sets are the primary source of questions for tests.

	\item \textbf {Taking Notes:}

The act of taking notes helps remember things to come back to later. I do not give out notes on lectures. If you are not in class, it is your responsibility to get the information from a classmate.

        \item \textbf {Assignments and Labs:}

There will be 3 graded assignments during the course. They will be assigned 1-2
weeks prior to their due date.  Assignments must be turned in on the Canvas
website. Assignments are not accepted late for full credit. If you turn in an
assignment late, it is kept until the end of the semester. At that time it will
be considered for a maximum of 50\% credit.  Assignments must be turned in on
Canvas to be considered for credit.

\end{itemize}

\textbf {Group Work:}

\hspace{3mm}

If you choose to work with other students on in-class exercises or assignments,
include contributors' names, login, day and time in your Canvas submission. All contributors will receive the same grade. 

\hspace{3mm}

\textbf{Getting Help:}

\hspace{3mm}

This course will use a Slack channel to get questions and answers. Borderline
grades will receive extra-credit points with active postings to the Slack
channel.  Also, this will be accessible between course sessions. The ACRC lab
(3rd floor Batmale hall) also has tutors who are willing to help with basic
Linux.

\hspace{3mm}

\textbf{Hills Accounts:}

\hspace{3mm}

Within a few days of completing your enrollment for this class, an account will
be created for you on our school Linux server (hills). If you are adding, it is
important that you complete your add as soon as possible. Delaying your add to
the class will not be considered an excuse for turning in assignments late. If
you had a hills account last semester, your account and password will remain
the same. If you are getting a new hills account, your account name will be the
same as your CCSF e-mail account (if you did not "opt-out"). Your initial
password is formed from your birth date - combining the first three letters of
the month (lowercase) with the two-digit day and the two-digit year followed by
a period and the first two characters of your login; e.g., if your e-mail
account is scharo11@mail.ccsf.edu, and your birthday is Apr 14 1986 your hills
login would be scharo11 and your initial password would be apr1486.sc You
should immediately change your initial password.

\hspace{3mm}

\textbf{Access to Hills:}

\hspace{3mm}

You can access hills either from a computer in the ACRC in Batmale Hall or
remotely using ssh. If you access hills from the ACRC you should use the linux
machines near the rear exit (see the next section). You may also login from a
Windows system, but you must first login to the ACRC Windows network. If you
wish to do this, you should take an orientation during the first week of class.
You can access hills remotely using ssh. Do not use telnet. The particulars of
remote access are your responsibility. The ACRC holds a series of three
orientation classes on hills and on their Windows network. A schedule is posted
in the ACRC. If you are new to hills or the ACRC you should consider attending
these sessions.

\hspace{3mm}

\textbf{Access to Linux Machines (Springfield Cluster)}

\hspace{3mm}

By enrolling in this course you will have an account on the linux machines.
These accounts will be created a week or so into the semester as announced in
class. The account name and initial password follow the same pattern as your
hills account. The linux machines are divided between those in the ACRC and
those in the linux classroom. They all share a common set of logins and a
common exported file system for home directories. Once on linux, you can use
ssh at the command-line to log in to hills to access the class public data
files.  For security reasons the linux machines are only accessible through
hills or another local machine. They are not registered via DNS. If you want to
reach a linux system from off-campus, you must login to hills and ssh using the
IP address of a linux machine. These IP addresses are taped to the linux
machines in the linux area of the ACRC. You should visit it and make a note of
them.

% Course Outline

\hspace{3mm}

\textbf {\large Tentative Course Outline}:

The weekly coverage might change as it depends on the progress of the class.

\begin{table}[h!]
  \normalsize % The size of the table text can be changed depending on content. Remove if desired.

\begin{tabular}{ | c | c | }
\hline
\textbf{Due Date} & \textbf{Homework Schedule} \\
\hline
1/22 (M) & \begin{minipage}{.85\textwidth}
\begin{itemize} \itemsep-0.4em
        \vspace{1mm}
        \item Logging in: console and ssh
        \item Basic commands; script environment
        \item Assignment 1
        \item File structures
        \item SCP/SFTP
        \vspace{1mm}
\end{itemize}
\end{minipage} \\
\hline
1/29 (M) & \begin{minipage}{.85\textwidth}
\begin{itemize} \itemsep-0.4em
        \vspace{1mm}
        \item Wildcards
        \item Basic commands for file manipulation
        \vspace{1mm}
\end{itemize}
\end{minipage} \\
\hline
2/5 (M) & \begin{minipage}{.85\textwidth}
\begin{itemize} \itemsep-0.4em
	\vspace{1mm}
        \item 2/7 Assignment 1 due
        \item File utilities 
	\item Command line editing utilities
	\vspace{1mm}
\end{itemize}
\end{minipage} \\
\hline
2/12 (M) & \begin{minipage}{.85\textwidth}
\begin{itemize} \itemsep-0.4em
	\vspace{1mm}
        \item Assignment 2
        \item Quiz 1 IN CLASS
        \item File Permissions
	\vspace{1mm}
2/19 (M NO CLASS) & \begin{minipage}{.85\textwidth}
\begin{itemize} \itemsep-0.4em
	\vspace{1mm}
        \item 2/21 Assignment 2 Due
	\vspace{1mm}
\end{itemize}
\end{minipage} \\
\hline
2/26 (M) & \begin{minipage}{.85\textwidth}
\begin{itemize} \itemsep-0.4em
        \vspace{1mm}
        \item 2/26 Quiz 2 IN CLASS
        \item Assignment 3
        \vspace{1mm}
\end{itemize}
\end{minipage} \\
\hline
3/5 (M) & \begin{minipage}{.85\textwidth}
\begin{itemize} \itemsep-0.4em
	\vspace{1mm}
        \item Filters
        \item Basic Reg Ex
	\vspace{1mm}
\end{itemize}
\end{minipage} \\
\hline
3/12 (M) & \begin{minipage}{.85\textwidth}
\begin{itemize} \itemsep-0.4em
        \vspace{1mm}
        \item Advanced Reg Ex
        \item 3/14 Assignment 3 Due
        \item 3/14 Final Exam (Comprehensive)
        \vspace{1mm}
\end{itemize}
\end{minipage} \\
\hline
\end{tabular} 
\end{table}

\end{document}
